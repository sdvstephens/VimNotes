\lipsum[2]

\section{Vector field}
It's equivalent with the concept of `Class' in the programming language C. \\
\diagram 

\begin{align}
    \odd{f}{\tau} &= \odd{f(x(\tau))}{\tau} \\
    &= \pdd{f}{x^a} \notate{\boxed{\odd{x^a}{\tau}}}{1}{\equiv v^a \longrightarrow \text{velocity of the Observer}} \\
    \boxed{\odd{f}{\tau}} &= v^a \pdd{f}{x^a} \label{df/dtau_2.8}
    % \hphantom{\equiv v^a \longrightarrow \text{velecity of the Observer}}
\end{align}
If the coordinate is changed, $x \longrightarrow x'$:
\begin{align}
    \boxed{\odd{f}{\tau}}  &= \odd{f(x'(\tau))}{\tau} \\
    &= v'^b \pdd{f}{x'^b}
    % \boxed{v'^b &= p} 
\end{align}
Since we insist that $x \longleftrightarrow x'$:

\begin{DispWithArrows}[groups, tikz={blue, font={}}]
    v'^b \pdd{f}{x'^b} &= v^a \left( \pdd{f}{x^a} \right) \\
    &= v^a \left( \pdd{x'^c}{x^a} \pdd{f}{x'^c} \right) \Arrow{As $c$ is a dummy index} \\
    &= v^a \left( \pdd{x'^b}{x^a} \pdd{f}{x'^b} \right) \\
    \implies\, \Aboxed{v'^b &= v^a \pdd{x'^b}{x^a}} \label{v'b_2.14}
\end{DispWithArrows}
\eqref{v'b_2.14} is the transformation property of vector components.

From the eqn \eqref{df/dtau_2.8}:
\begin{equation}
    \odd{f}{\tau} = v^a \pdd{f}{x^a} = \hat{\left(v^a \pdd{}{x^a}\right)} f = \hat{V} f 
\end{equation}

\begin{gather}
    v = v_a \hat{e}_x \\
    \notate[X]{\boxed{\hat{e}_x = \pdd{}{x^a}}}{1}{\text{Coordinate basis}}
\end{gather}

\begin{gather}
    V = V_a \hat{e}_a = V'_b \hat{e}_{b'} \\
    \text{Here, } \hat{e}_a = M_a^b \hat{e}_{b'} \\
    \pdd{\,\Smiley}{x^a} =  \pdd{x'^p}{x^a} \pdd{\,\Smiley}{x'^p} \qquad \text{[ \Smiley is just a placeholder ]}
\end{gather}

\begin{takeaway}[What are vector fields?]
    Vector fields are just 1st order differential operator. \\
    They are directional derivatives (hence operators) $D_v = (\vec{v}\cdot\nabla)$
\end{takeaway}

\section{Covariant derivative}
\subsection{Why do regular partial derivatives fail?}
For scalar: \\
Scalar is invariant under a coordinate transformation. 

\begin{equation}
    \begin{tikzcd}
      f \arrow[d, ]\arrow[r, ] & f' \arrow[d, ] \\
      \del_a f(x) \arrow[r, ] & \del'_a f(x') \mathrlap{{} = \underbrace{\pdd{x^b}{x'^a}}_{M^{-1}}\del_b f(x)}
    \end{tikzcd}
    \hphantom{\pdd{x^b}{x'^a}}
\end{equation}

So, there's no problem with tensor properties.

For vector: 
\begin{equation}
    \begin{tikzcd}
      V^a \arrow[d, ]\arrow[r, ] & V'^a \arrow[d, ]  \mathrlap{ {} = \pdd{x'^a}{x^m} V^m } \\
      \del_a V^a \arrow[r, ] & \del'_a V'^a \mathrlap{ {} = \del'_b \left( \pdd{x'^a}{x^m} V^m \right) }
    \end{tikzcd}
    \hphantom{\pdd{x'^a}{x^m} V^m}
\end{equation}
\begin{align}
    \del'_a V'^a &= \del'_b \left( \pdd{x'^a}{x^m} V^m \right) \\
    &= \pdd{x^n}{x'^b} \del_n \left( \pdd{x'^a}{x^m} V^m \right) \\
    &= \pdd{x^n}{x'^b} \pdd{x'^a}{x^m}  \del_n V^m +  \notate[X]{\pdd{x^n}{x'^b} \left( \pdd{^2 x'^a}{x^n \del x^m} V^m \right)}{1}{\parbox[t]{2in}{This term spoils the tensor \\ transform properties}} \\ \nonumber
\end{align}
\bigskip 
So, the regular partial derivatives of vectors do not transform like tensors. They don't follow the tensor properties. \\
A covariant derivative has to transform covariantly, aka `tensorly'!

We have to define: 
\begin{equation}
    \nabla'_b V'^a = \pdd{x^n}{x'^b} \pdd{x'^a}{x^m} \nabla_n V^m 
\end{equation}

\lipsum[2]