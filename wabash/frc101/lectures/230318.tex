Before starting GR, let's discuss classical field theories. Let's go through Yang-Mills original paper \cite{yang1954conservation}. 

\section{Yang-Mills Field Theory}

\lipsum[3]

\begin{center}
    \begin{forest}
        [\underline{Symmetry}
            [Global Symmetry\\ (unphysical)] [Local Symmetry]
        ]
    \end{forest}
\end{center}

\lipsum[2]

Now, considering the derivative $D_a$ as:
\begin{equation}
\begin{tikzcd}
  \Psi \arrow[d, "D_a"]\arrow[r, "U"] & \Psi' \arrow[d, ] \mathrlap{{} = U\Psi}\\
  D_a \Psi \arrow[r, "U"] & (D_a\Psi)' \mathrlap{{} = U(D_a\Psi)}
\end{tikzcd}
\hphantom{{} = U(D_a\Psi)}
\end{equation}
\begin{equation} \text{Here, } D_a \Psi' = U D_a \Psi \end{equation}
The derivative that satisfies this property is called a `covariant derivative'. Why covariant? Because it's a derivative that transforms covariantly. In GR, YM theory, gauge theory, everywhere - we define appreciate covariant derivatives. \details 

The definition of $D_a$ on $\Psi$: 
\begin{equation}
 D_a \Psi = \notate{\del_a}{3}{\text{regular derivative}}\Psi + i\notate{g}{2}{\text{coupling constant}} \notate{A_a}{1}{\text{transport}} \psi \label{DaPsi_1.16}
\end{equation}

\lipsum[2]

\begin{remark}
 Claim: This is true for any 1-parameter transformation in Lie groups.
\end{remark}
\begin{figure}[H]
    \centering
    \incfig{0.3}{230318-5}
    \caption{Lie group is a manifold. In n-dimensional space, to go from point A to B, we need to change n-parameters. But we don't have to do that, rather we can choose to go from A to B in 1 curve. By 1-parameter, we mean we can go through a curve by a single parameter. (1 smooth curve transfer motion). There are examples of certain groups where it's not possible, for example, $SL(2,\RR)$. Check Arnold \cite{arnold2013mathematical} for details.}
    \label{fig:manifold1}
\end{figure}

\lipsum[2]